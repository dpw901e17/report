\section{Graphics Hardware}\label{sec:graphics_hardware}


\begin{sectionmeta}
	
	This section will introduce the \gls{GPU} from a hardware standpoint. 
	First the overall concept of a \gls{GPU} will be described - what it is, what it does, and how it achieve its purpose.
	
	\cite{intro_to_gpu_arch} describes the architecture and components of a \gls{GPU} as the result of three ideas.
	These ideas will be presented here.
	Different terminology for the individual components will be presented as they are used by the \gls{GPU} vendors, NVidia and AMD. 
	
\end{sectionmeta}


\subsection{GPU as a Concept}
The \gls{GPU} has been developed with a specific domain in mind, as opposed to the CPU, which is for general purposes. 
The domain of the \gls{GPU} was originally only image manipulation -- a field, where a \gls{SIMD} architecture has proven useful \todo{citation needed}.
%In recent years, however, there has been a focus on using the \gls{GPU} for a broader spectrum of applications, the socalled \gls{GPGPU}. 

\fig{figures/graphics_hdw_cpu_style_core}{\gls{CPU} style core \cite[p. 14]{intro_to_gpu_arch}}{cpuStyleCore}{1}

Figure \ref{fig:cpuStyleCore} shows a visual representation of a \gls{CPU} style core. 
The red boxes "Out-of-order control logic", "Fancy branch predictor" and "Memory pre-fetcher" all have to do with predicting/preventing stalls in the \gls{CPU}.
These features are not too important for the \gls{GPU}, since its main focus is throughput \todo{citation needed}. 
Furthermore, a big cache would limit the amount of cores a single chip could hold, so this is not desirable for a \gls{GPU} either.
The remaining components -- Fetch/Decode, \gls{ALU} and Execution Context -- is described below.

\paragraph{The Fetch/Decode Component} handles retrieving data from memory and storing it in the Execution Context.

\paragraph{The \gls{ALU}} performs the actual computations on the fetched data. Any temporary variables or conditions (when handling branches) are stored or retrieved from the Execution Context.

\paragraph{The Execution Context} contains local data, e.g. variables and conditions, needed to perform the current computation.

\subsection{\acs{CPU} Style Core}
As previously described, there are components in the \gls{CPU} style core which are not needed for the \gls{GPU} to achieve a high throughput.
So the first idea presented by \citet{intro_to_gpu_arch} is to "slim down" the core by getting rid of these components.
The result of slimming down the core can be seen in Figure \ref{fig:twoSlimCores}.

\fig{figures/graphics_hdw_two_cores}{Two slimmed down cores \cite[p. 16]{intro_to_gpu_arch}}{twoSlimCores}{1}

\fig{figures/graphics_hdw_shader_code}{A closer look at the shader code in \ref{fig:twoSlimCores} - \citet[p. 22]{intro_to_gpu_arch}}{shaderCode}{0.5}

Figure \ref{fig:twoSlimCores} presents two slimmed down cores running two fragments in parallel. 
Each core runs the same code, but since the contents of the Execution Contexts are different, we achieve the desired \gls{SIMD} effect; the two fragments are processed by the same code (Single Instruction), but the code refers to data through registers in the Execution Context, which is different for the two cores (Multiple Data).

Figure \ref{fig:shaderCode} displays the shader code run on fragment 1 and 2 from \ref{fig:twoSlimCores}.

\subsection{Slimmed Down Core}
Fetching data and instructions are a relatively time-costly activity for the core \todo{citation needed}.
The second idea, to further the throughput of the \gls{GPU}, is to let multiple \glspl{ALU} share a single Fetch component.
This way, the component need to retrieve more information at a time, less times, which is not as costly as retrieving small bits of information (data and instructions) for a single \gls{ALU}.
This means the actual instruction which needs to be run on each \gls{ALU} can be fetched only once per core instead of once per \gls{ALU}.

\fig{figures/graphics_hdw_shared_fetch}{Fetch component shared by eight \glspl{ALU}. Note the instructions need to change to use vector operations on vector data as well \cite[p. 24]{intro_to_gpu_arch}}{sharedFetch}{1}

Figure \ref{fig:sharedFetch} shows an example of a core with eight \glspl{ALU} sharing a single instruction stream (i.e. Fetch component).
Since the instructions need to be carried out on a vector of data (each element in the vector corresponding to data for a single \gls{ALU}).
The instructions need to reflect this change from single data to vector of data -- hence the change from "mul" and "madd" from \ref{fig:twoSlimCores} to "VEC8\_mul" and "VEC8\_madd" in Figure \ref{fig:sharedFetch} in the shader.

Since the core now contains multiple contexts, the Execution Context component will now be referred to as the \textbf{Shared Context component}.

\subsection{Hiding Stalling}
The final thing \glspl{GPU} do to achieve a high throughput is to hide stalling by storing multiple contexts for different fragments on a single core. 
This allows the core to switch which fragment it works on once the current fragment stalls.
Stalling occurs when the processing of a fragment group is dependent on another fragment group which is not done processing itself (recall the \glspl{ALU} each work on its own fragment, as per the slimmed down cores.  The fragments being worked on by the \glspl{ALU} at the same time constitutes a fragment group).
Because switching between contexts is faster than waiting for the context to come out of a stall, the amount of time the \gls{GPU} takes to perform a job is lessened by the hiding of stalls; the latency of the \gls{GPU} caused by stalling has been hidden, which is why this process is also known as latency hiding.

This latency hiding through interleaving execution of groups of fragments is done because the first idea \todo{ref?} stripped the core of the means the \gls{CPU} uses to hide stalling.

\fig{figures/graphics_hdw_hiding_stalls_1}{The shared context data is split up to match the (here four) different fragment groups \cite[p. 35]{intro_to_gpu_arch}}{hidingStalls1}{1}

\fig{figures/graphics_hdw_hiding_stalls_2}{When one fragment stalls, the \gls{GPU} switches to another stored context and continues on another fragment \cite[p. 37]{intro_to_gpu_arch}}{hidingStalls2}{1}

The figures \ref{fig:hidingStalls1} and \ref{fig:hidingStalls2} shows the \gls{GPU} latency hiding process.
First the Shared Context component is divided into the number of fragment groups the core shall be able to process -- four in this example.
Then the processing of the first fragment group is begun. 
Once it stalls (or completes), the processing of the next fragment group can begin.
The idea is: Once all fragment groups have been cycled trough, the cause of the stall of the first fragment group has been resolved.

\subsection{Branching in \glspl{GPU}}
One last thing worth noting from \cite{intro_to_gpu_arch} is how \glspl{GPU} handle branching.

\fig{figures/graphics_hdw_branching}{How \glspl{GPU} handle branching \cite[p. 29]{intro_to_gpu_arch}}{branching}{1}

Figure \ref{fig:branching} shows how branching is handled within a single core: The instructions for both branches are executed (on different \glspl{ALU}), and later the correct result will be chosen.

\subsection{Contemporary Graphics Cards}
This subsection will describe the architecture of  contemporary NVIDIA, Intel and AMD graphics cards and compare the terminology used by these vendors to the terminology presented in this section.
The selected graphics cards are:
\begin{itemize}
	\item NVIDIA GeForce GTX 1060
	\item Intel HD Graphics 4600
	\item AMD Sapphire Radeon R9 280 3GB GDDR5
\end{itemize}

\subsubsection{NVIDIA}
The NVIDIA GeForce GTX 1060 graphics card is build with the NVIDIA Pascal architecture \cite{nvidia_gtx_1060}.

\fig{figures/graphics_hdw_pascal_sm}{A Streaming Multiprocessor in the Pascal architecture. - \cite{nvidia_tesla_p100} p. 13}{pascalSM}{1}

Figure \ref{fig:pascalSM} shows a \gls{SM} in the Pascal architecture.
To translate the terminology from NVIDIA to what has been presented in this section, the small rectangles labelled "Core", "DP Unit", "LD/ST" and "SFU" all fall under the category \gls{ALU} as presented earlier (technically, each of these components consists of multiple \glspl{ALU}).
These are in NVIDIA terminology collectively referred to as "CUDA Cores".

The blue caches, textures and buffers are all part of the Shared Context component.

The Fetch/Decode component is not visible in figure \ref{fig:pascalSM}.

From the above translation of terminology, the \gls{SM} seems to map quite well onto what has been presented as the "core".
However, it is obvious that there exist a further grouping inside the \gls{SM}: The left-hand side and the right-hand side.
Even though the two sides share an instruction stream and some memory, they are still two distinct sides.
These sides are what is referred to as "Warps" in NVIDIA terminology.

\fig{figures/graphics_hdw_pascal_gpu}{A 60 \gls{SM} units Pascal GP100 GPU - \cite{nvidia_tesla_p100} p. 10}{pascalGPU}{0.8}

Figure \ref{fig:pascalGPU} shows a complete GP100 GPU in the Pascal Architecture.
Two \glspl{SM} are grouped into one \glspl{TPC}, and five \glspl{TPC} are grouped together into a \gls{GPC}. \\

It should be noted that the NVIDIA GeForce GTX 1060 uses a GP106 chip, not the GP100 chip shown in \ref{fig:pascalGPU}. 
The main difference between the two chips is the number of CUDA Cores: The GP100 has 3840 (60 \glspl{SM}, each with 64 CUDA Cores), and the GP106 only have 1280 (20 \glspl{SM}, each with 64 CUDA cores).

We were not able to find a whitepaper or similar report from reliable sources which detailed the GP106 chip, so that is why it is not used as the example. 


\subsubsection{AMD}
For this project we have a AMD Sapphire Radeon R9 280 3GB GDDR5 \gls{GPU} at our disposal. 
It uses the GCN 1.0 Architecture.
GCN stands for Graphics Core Next, it is the first architecture AMD developed that would allow to use the \glspl{GPU} which were built on it as \glspl{GPGPU}.
Compared to the previous architecture AMD \glspl{GPU} were built with (VLIW4) GCN is better at handling threads with less cycles going to waste.
GCN goes chronologically through to check which threads requires the other to finish beforehand, and then decides in which order the threads should be executed.
This results in less cycles needed in order to finish the tasks a thread can have.
\todo{I found a white paper that explains AMD GCN much further in detail, but cannot find anything that specificially mentions the graphics cards. Should I go more in depth about the architecture?}

\subsubsection{Intel}
Traditionally Intel only functioned as a \gls{CPU}, however overtime it was realized there was some performance gain to be had by incorporating inbuilt \glspl{GPU} into Intels \glspl{CPU}.
Intel introduced its first series of integrated graphics processors in 2010 and have since then kept releasing newer and more powerful once than the last generation. 
These \glspl{CPU} are called Intel HD Graphics. 
In 2013 Intel introduced Intel Iris Graphics and the Intel Iris Pro Graphics series in 2013 which also integrated a graphics processor. 
They are better than their HD Graphics counterpart, they distinguish form their HD Graphics counterpart as they were the first series to include embedded DRAM into their \glspl{CPU}. 

\todo{figure out why they are more powerful in terms of throughput}
Additionally, Iris \glspl{CPU} are made for desktop computers, whereas HD Graphics \glspl{CPU} are made for laptop computers.
Regardless of which Intel \glspl{GPU} integrated \glspl{CPU}, the inbuilt processing unit has the advantage of consuming less power than a Nvidia or AMD \glspl{GPU}.
This comes at the cost of the \gls{GPU} not being able to give as much throughput as Nvidia or AMD. 
However, there are technologies like Nvidia Optimus that tries to exploit this. 
If some GPU command does not demand much \gls{GPU} throughput, it can automatically make use of the Intel inbuilt \gls{GPU} instead and thus save power.
They have support for both Direct3D and OpenGL. Depending on the version, some are also supporting Vulkan.

\todo{specify the Intel GPU’s we have at our diposal, and how they work.}
%HD Graphics 520 – Skylake %(claus’ cpu)
%HD Graphics 4000 – Ivy Bridge (michaels cpu)
%HD Graphics 4600 – Haswell &(anders’ cpu)
%HD Graphics Family (brandborgs cpu)

