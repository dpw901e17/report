\chapter{Introduction}\label{ch:introduction}
In the early days of consumer electronics, computer visuals were mainly displayed as text.
Yet, with the combined rise of microcomputers and \glspl{GUI} in the 80's, graphics and visuals became a hot topic for both developers and consumers.  
In the 90's, highly parallel graphics hardware in the form of GPUs were introduced to render 3D models in real time.
This piece of hardware has become a stable in modern consumer computers, either as individual components or integrated the CPU.

%Computers and applications are no longer the niche market it was in its early days, and from that comes a need for intuitive interfaces for users. 
%Around the 1970s the first computers with \glspl{GUI} started to appear to improve the user experience.
To ease development of graphical applications, graphics \glspl{API} have been created. 
These allow developers to work with the GPU at a more abstract level, through an interface.
%But to develop these \glspl{GUI}, developers use a lot of time working with the graphics \glspl{API}.
Developers have used these \glspl{API} to create impressive applications, such as modern video games, which render graphics in real time at 60 frames per second, as well as the graphically demanding \glspl{GUI} of modern operating systems.

%As hardware improves, so must the \gls{API} to better use the improvements.
%Another function of the \gls{API} is to make it easier for the developers to use the hardware so improvement in that area is also interesting for the \gls{API} developers.
Graphics \gls{API} have to evolve alongside the modern CPU and GPU architectures, to take advantage of new features. 
Yet, academic research into the evaluation of \glspl{API} and how to improve them is limited.
Even if the newer \glspl{API} supports operations that improve performance, if the feature is too hard to use most developers will ignore it.

With the release of the most recent graphics \glspl{API}; Vulkan and Direct3D 12, there has been a push towards a more low level API design, which gives the programmer more control over the graphics pipeline.
Much of the functionallity that was previously handled by the driver, is now handled by application developers. 
This has some advantages, which will be described in later sections. \todo{insert sections referencer.}
But the argument that lower levels of abstraction means better \glspl{API} would suggest that writing in binary or assembly would give the best \glspl{API}.

In this report, we are instested in how the newer \glspl{API} lower their abstraction-level, and what advantages and disadvantages that approach has both on performance and programability.
The aim is to generate insight into the graphics \glspl{API}; Direct3D 12 and Vulkan, and to generate tools for future developers that can be used to evaluate and improve their \glspl{API}.

Direct3D 12 and Vulkan are chosen, because they are released close to eachother (2015 and 2016 respecively), and have some overlapping ideas on how to improve graphics \glspl{API}, such as better multithread support on the CPU.
The predesessors to these \glspl{API} (Direct3D 11 and OpenGL 4.5) are some of the most broadly used graphics \glspl{API} for realtime rendering, and we suspect that Direct3D 12 and Vulkan will have similar success.

\todo[inline]{insert overgang til næste sektion}

\section{The Short History of Graphics \acs{API}s}\label{sec:short_history}
\begin{sectionmeta}
	This section gives an overview of the evolution of computer graphics with the focus being on the \glspl{API}.
	First section will look at graphics before graphics \glspl{API}, the example given will be that of a Commodore 64.
	After that, the early competition for getting the graphics \gls{API} market is described.
	Lastly, we look at some of the more modern improvement that is made to the modern \glspl{API}.
	
	In this section, the term 3D accelerator is used interchangeably with graphics processor or \gls{GPU}, because this is the historically accurate term. 
	They all refer to hardware designed to render 3D graphics.
\end{sectionmeta}

\subsection{Early Days of Consumer Graphics}

In the early 80s, dedicated hardware for rendering images started to appear.
Back then, video cards were not expected to have their own video memory, and instead shared memory with the \gls{CPU} \cite{wikipedia????shared}.

This was a problem because of the limited memory avaliable to the computer.
For instance, the Commodore 64 had 64 kilo bytes of memory, and supported a 320 by 200 pixels screen \cite{commodore1983commodore}.
This meant that storing a full screen bitmap with four colors would take $$320\times 200\times 2 \text{ bits} = 128,000 \text{ bits} = 16,000 \text{ bytes of memory.}$$
One quarter of the total 64 kilo bytes of memory for storing the bitmap.
If the image is using 16 colors then half of the Commodore 64 memory would be used, which would leave little space for the program and I/O, which was memory mapped \cite{commodore1983commodore}.

The Commodore 64 used the MOS Technology VIC-II graphics chip \cite{commodore1983commodore}.
The chip supports a 320 by 200 pixels video resolution, 16 colors, and 16 kB memory for screen, character, and sprites \cite{commodore1983commodore}.\todo{Not sure if this memory is also shared by the CPU.}
It could handle a maximum number of eight sprites on each horizontal line of pixels.
Each sprite was a 24 by 21 pixels, 2 color bitmap with one of colors being transparent \cite{commodore1983commodore}.

But how could it fit 320 by 200 screen with 16 colors into 16 kB of memory?
Using conventional methods, this was not possible.

Instead, the Commodore 64 used a technique known as color cells \cite{commodore1983commodore}, which were smaller screen segments that used a limited color palette.
This technique was also used by the Nintendo Entertainment System. 
The Commodore 64 had two modes for displaying color cells: High resolution mode, where all 320 by 200 pixels were usable, but only two colors; and multi-color mode, where every second pixel was a repeat of the previous \cite{commodore1983commodore}.
The benefit is that in multi-color mode you can use four colors instead of two.
This was popular for games since color was more important than resolution \cite{bogdan2014games}.
Regardless of which mode is used, the screen would use 8 kB of memory and palettes would use 1 kB memory, the rest of the memory was used for sprites \cite{commodore1983commodore}.

Sprites were a separate system that could be drawn on top of the bitmap.
The Commodore 64 supported up to eight sprites \cite{commodore1983commodore}. \todo{maybe visualize what part of the memory maps to what}

Other systems such as the Nintendo Gameboy used similar methods with different restrictions.
The Nintendo Gameboy supported 40 sprites, but they had to be 8 by 8 pixels, and only 10 could be on a single scanline \cite{nintendo1999gameboy}. 
The screen was also smaller, 160 by 144 pixels, and always supported four colors, that had to be the infamous four shades of green \cite{nintendo1999gameboy}.

To program graphics on the Commodore 64, the programmer would write to and read from the 47 graphics registers in the Commodore 64s address range 53,248 to 53,294 \cite{commodore1983commodore}.
As an example, if you were to put a sprite into the top left corner of the screen after inserting the sprite data into the memory you would write the following BASIC commands; 

\begin{lstlisting}[caption={Small BASIC program that sets the coordinate of sprite 0 to (0,0) (line 1-3), sets the color to green (line 4), and displays it (line 5).}]
10 POKE 53248,0
20 POKE 53264,0
30 POKE 53249,0
40 POKE 53287,5
50 POKE 53269,1
\end{lstlisting}
This is fairly verbose, both because it is written in tha BASIC language, and because you had to write to specific hardcoded registers.
This was the way it was done before graphics \glspl{API} became prominent.

It was first with the rise of 3D rendering that graphics \glspl{API} became a prominent part of the toolset available to developers.
And advances in 3D rendering had a benificial effect of 2D rendering effeciency.


%%%%%%%%%%%%%%%%%%%%%%%%%%%%%%%%%%%%%%%%%%%%%%%%%%%%%%%%%%%%%%%%%%%%%%%%%%%%%%%%%%%%%%
\subsection{The Rise of Graphics \acs{API}s} 

In January of 1992 Silicon Graphics, Inc. released the first version of OpenGL, which tried to streamline the process of 2D and 3D graphics development \cite{segal1994opengl}.
It was based on their own proprietary graphics libary: \gls{IRIS GL}.
The first version of OpenGL was far from perfect and some of its flaws, such as no texture objects \cite{kronos????history}.

When RenderMorphic's Reality Lab started to gain traction, Microsoft bought them in Febuary of 1995 and it became the core for Microsoft's own 3D rendering \gls{API} \cite{1997crushed}; Direct3D.
The first version of Direct3D shipped in June 1996 with DirectX 2.0 \cite{wikipedia????directx}. 

Around the same time 3dfx's Brian Hook wrote Glide, a graphics API for the 3dfx Voodoo Graphics Accelerator.
%Glide was inspired by OpenGL, but focused on features that were useful for realtime 3D rendering.
Because of the ease of programming to Glide, and the popularity of the Voodoo graphics accelerators, Glide became the dominating graphics \gls{API} in the late 1990's.

Glide's success was in part made possible by Microsoft's involvement in the Talisman project, which took some of resources from the DirectX team.
The Talisman project was a new 3D architechture that could reduce the memory bandwidth needed by applying tiled rendering \cite{torborg1996talisman}.
But as other 3D accelerators became more efficient and cheaper, Microsoft cancelled project Talisman and reverted their focus to Direct3D \cite{wikipedia????talisman}.
The market share of the Windows operating system gave Microsoft the consumer base that the hardware specific \glspl{API} lacked, so developers soon started to focus their attention on Direct3D.

John \citet{carmack1996plan} of Id Software critized Direct3D for being too verbose, after trying to port Quake to OpenGL and Direct3D.
So there were some resistance to Direct3D's market dominance, so some developers insisted on using OpenGL.
And so the OpenGL and Direct3D became the dominant competitors in the 3D rendering 

\subsection{Modern Graphics \acs{API}s}

The early days of 3D programming was still very verbose and inflexible. 
Fixed function pipelines were used, which meant you could not change how the graphics pipeline worked \cite{davidovic2014fixedfunction}.

In November 2000, Microsoft released their first version of \gls{HLSL} together with DirectX 8 \cite{wikipedia????directx}.
\Gls{HLSL} is a language for small programs that runs on the \gls{GPU} instead of the \gls{CPU}.
It is designed as a high level, C-like language with extenstions to include some of the most used types such as vectors and matrices \cite{microsoft????hlsl}.
It also served as a way to decide which calculations should be performed in which render stages in the pipeline.
This introduced some flexibility to the graphics pipeline \todo{Should introduce the graphics pipeline earlier}.

OpenGL followed with their own shader language \gls{GLSL} in April 2004 \cite{wikipedia????opengl}. \todo{haven't introduced shaders before}
Like \gls{HLSL}, \gls{GLSL} is a C-like language with some extensions to better support operations common to graphics rendering.

\vspace{1em}

\noindent
Because OpenGL and Direct3D is in direct competition with eachother, and operate on the same hardware, once a technology, or tool has been developed for one, the other usually follows soon after.
Therefore, we look at the general graphics \gls{API} improvements in the following sections.

\vspace{1em}

\noindent
Shaders have since then been improved with more features, such as changing model details with the geometry shader, and generating model details via a tessellation stage.

The geometry shader is placed between the vertex and fragment shader and operated on primitives; points, lines, or triangles, and returned zero or more primitives for rasterizing.
Because the geometry shader could see multiple verticies at once, developers were excited for the oppotunities this would give \cite{kronos????geometry, microsoft????geometry}.
A good example of using the geometry shader is particle effects, where the specifics of the four corners of the mesh is less important for the developer than the position of the particle.
So instead, the developer can send the points of the particles to the \gls{GPU}, and have the geometry shader generate the corners in parallel.
One big issue with geometry shaders is that their performance overhead is so large that naïvely using it would often hurt performance rather than improving it.

The next change to the rendering pipeline was the introduction of a tessellation stage.
This stage involves two shaders; a hull shader and a domain shader.
It occurs after the vertex shader, but before the geometry shader.
The tessellation stage generates more details to the model by adding even more vertices.
Where the vertices are placed and how many there are, is handled by the shaders.

Later on, tessellation became the big topic, which is way to generate model details on the \gls{GPU}.
The tessellation stage is a way achieve some of the things developers imagined was possible with geometry shaders, but was not feasible because of the performance overhead the shader had.

The most recent trend is to lower the abstraction level of the \glspl{API} to enable more customization.
This push can be seen in the most recent version of Direct3D; Direct3D 12, and the so called successor to OpenGL; Vulkan.
The lower level abstract enables the developer to customize the \gls{GPU} usage to their needs.
Removing this layer of abstraction also has the benefit of enabling multi-threaded \gls{GPU} calls, by eliminating the driver state.
The drawback is that these lower levels of abstraction are more verbose, and thus increases the complexity of even the simple task of drawing a single triangle on screen.

\section{Graphics Hardware}\label{sec:graphics_hardware}


\begin{sectionmeta}
	
	This section will introduce the \gls{GPU} from a hardware standpoint. 
	First the overall concept of a \gls{GPU} will be described - what it is, what it does, and how it achieve its purpose.
	
	\cite{intro_to_gpu_arch} describes the architecture and components of a \gls{GPU} as the result of three ideas.
	These ideas will be presented here.
	Different terminology for the individual components will be presented as they are used by the \gls{GPU} vendors, NVidia and AMD. 
	
\end{sectionmeta}


\subsection{GPU as a concept}
The \gls{GPU} has been developed with a specific domain in mind, as opposed to the CPU, which is for general purposes. 
The domain of the \gls{GPU} was originally only image manipulation - a field, where a \gls{SIMD} architecture has proven useful.
%In recent years, however, there has been a focus on using the \gls{GPU} for a broader spectrum of applications, the socalled \gls{GPGPU}. 

\fig{figures/graphics_hdw_cpu_style_core}{\gls{CPU} style core - \cite{intro_to_gpu_arch} p. 14}{cpuStyleCore}{1}

\ref{fig:cpuStyleCore} shows a visual representation of a \gls{CPU} style core. 
The red boxes "Out-of-order control logic", "Fancy branch predictor" and "Memory pre-fetcher" all have to do with predicting/preventing stalls in the \gls{CPU}.
These features are not too important for the \gls{GPU}, since it's main focus is throughput \todo{citation needed}. 
Furthermore, a big cache would limit the amount of cores a single chip could hold, so this is not desirable for a \gls{GPU} either.
The remaining components - Fetch/Decode, ALU and Execution Context - is described below.

\paragraph{The Fetch/Decode component} is responsible for retrieving data from memory and storing it in the Execution Context.

\paragraph{The \gls{ALU} component} performs the actual computations on the fetched data. Any temporary variables or conditions (when handling branches) are stored/retrieved from the Execution Context.

\paragraph{The Execution Context component} contains local data, e.g. variables and conditions, needed to perform the current computation.

\subsection{The first idea}
As previously described, there are components in the \gls{CPU} style core which are not needed for the \gls{GPU} to achieve a high throughput.
So the first idea presented in \cite{intro_to_gpu_arch} is to "slim down" the core by getting rid of these components.

\fig{figures/graphics_hdw_two_cores}{Two slimmed down cores - \cite{intro_to_gpu_arch} p. 16}{twoSlimCores}{1}

Figure \ref{fig:twoSlimCores} presents two such cores running two fragments in parallel. 
Each core runs the same code, but since the contents of the Execution Contexts are different, we achieve the desired \gls{SIMD} effect.

\subsection{The second idea}
Fetching data and instructions are a relatively time-costly activity for the core \todo{citation needed}.
The second idea to further the throughput of the \gls{GPU} is to let multiple \glspl{ALU} share a single Fetch component.
This way, the component need to retrieve more information at a time, less times, which is not as costly as retrieving small bits of information (data and instructions) for a single \gls{ALU}.

\fig{figures/graphics_hdw_shared_fetch}{Fetch component shared by eight \glspl{ALU}. Note the instructions need to change to use vector operations on vector data as well - \cite{intro_to_gpu_arch} p. 24}{sharedFetch}{1}

Figure \ref{fig:sharedFetch} shows an example of a core with eight \glspl{ALU} sharing a single instruction stream (i.e. Fetch component).
Since the instructions need to be carried out on a vector of data (each element in the vector corresponding to data for a single \gls{ALU}), the instructions need to reflect this change from single data to vector of data - hence the change from "mul" and "madd" from \ref{fig:twoSlimCores} to "VEC8\_mul" and "VEC8\_madd" in \ref{fig:sharedFetch} in the shader.

Since the core now contains multiple contexts, the Execution Context component will now be referred to as the \textbf{Shared Context component}.

\subsection{The third idea}
The final thing \glspl{GPU} do to achieve a high throughput is to hide stalling by storing multiple contexts for different fragments on a single core. 
This allows the core to switch which fragment it works on once the current fragment stalls.
Stalling occurs when the processing of a fragment group is dependent on another fragment group which is not done processing itself (recall the \glspl{ALU} each works on its own fragment as per idea number two - the fragments being worked on by the \glspl{ALU} at the same time constitutes a fragment group).
This latency hiding through interleaving execution of groups of fragments is done because the first idea \todo{ref?} stripped the core of the means the \gls{CPU} uses to hide stalling.

\fig{figures/graphics_hdw_hiding_stalls_1}{The shared context data is split up to match the (here four) different fragment groups. - \cite{intro_to_gpu_arch} p. 35}{hidingStalls1}{1}

\fig{figures/graphics_hdw_hiding_stalls_2}{When one fragment stalls, the \gls{GPU} switches to another stored context and continues on another fragment. - \cite{intro_to_gpu_arch} p. 37}{hidingStalls2}{1}

The figures \ref{fig:hidingStalls1} and \ref{fig:hidingStalls2} shows the \gls{GPU} latency hiding process.
First the Shared Context component is divided into the number of fragment groups the core shall be able to process - four in this example.
Then the processing of the first fragment group is begun. 
Once it stalls (or completes), the processing of the next fragment group can begin.
The idea is: Once all fragment groups have been cycled trough, the cause of the stall of the first fragment group has been resolved.

\subsection{Branching in \glspl{GPU}}
One last thing worth noting from \cite{intro_to_gpu_arch} is how \glspl{GPU} handle branching.

\fig{figures/graphics_hdw_branching}{How \glspl{GPU} handle branching. - \cite{intro_to_gpu_arch} p. 29}{branching}{1}

Figure \ref{fig:branching} shows how branching is handled within a single core: The instructions for both branches are executed (on different \glspl{ALU}), and later the correct result will be chosen.
 
\subsection{Contemporary Graphics Cards}
This subsection will describe the architecture of  contemporary NVIDIA, Intel and AMD graphics cards and compare the terminology used by these vendors to the terminology presented in this section.
The selected graphics cards are:
\begin{itemize}
	\item NVIDIA GeForce GTX 1060
	\item Intel HD Graphics 4600
	\item AMD Sapphire Radeon R9 280 3GB GDDR5
\end{itemize}

\subsubsection{NVIDIA}
The NVIDIA GeForce GTX 1060 graphics card is build with the NVIDIA Pascal architecture \cite{nvidia_gtx_1060}.

\fig{figures/graphics_hdw_pascal_sm}{A Streaming Multiprocessor in the Pascal architecture. - \cite{nvidia_tesla_p100} p. 13}{pascalSM}{1}

Figure \ref{fig:pascalSM} shows a \gls{SM} in the Pascal architecture.
To translate the terminology from NVIDIA to what has been presented in this section, the small rectangles labelled "Core", "DP Unit", "LD/ST" and "SFU" all fall under the category \gls{ALU} as presented earlier (technically, each of these components consists of multiple \glspl{ALU}).
These are in NVIDIA terminology collectively referred to as "CUDA Cores".

The blue caches, textures and buffers are all part of the Shared Context component.

The Fetch/Decode component is not visible in figure \ref{fig:pascalSM}.

From the above translation of terminology, the \gls{SM} seems to map quite well onto what has been presented as the "core".
However, it is obvious that there exist a further grouping inside the \gls{SM}: The left-hand side and the right-hand side.
Even though the two sides share an instruction stream and some memory, they are still two distinct sides.
These sides are what is referred to as "Warps" in NVIDIA terminology.

\fig{figures/graphics_hdw_pascal_gpu}{A 60 \gls{SM} units Pascal GP100 GPU - \cite{nvidia_tesla_p100} p. 10}{pascalGPU}{0.8}
	
Figure \ref{fig:pascalGPU} shows a complete GP100 GPU in the Pascal Architecture.
Two \glspl{SM} are grouped into one \glspl{TPC}, and five \glspl{TPC} are grouped together into a \gls{GPC}. \\

It should be noted that the NVIDIA GeForce GTX 1060 uses a GP106 chip, not the GP100 chip shown in \ref{fig:pascalGPU}. 
The main difference between the two chips is the number of CUDA Cores: The GP100 has 3840 (60 \glspl{SM}, each with 64 CUDA Cores), and the GP106 only have 1280 (20 \glspl{SM}, each with 64 CUDA cores).

We were not able to find a whitepaper or similar report from reliable sources which detailed the GP106 chip, so that is why it is not used as the example. 


\subsubsection{Intel}


\subsubsection{AMD}

\section{Graphics APIs}\label{sec:graphics_apis}
\begin{sectionmeta}
In this section we should introduce, why APIs are important to use, the basics on what graphics APIs are and finally what direction moderns APIs are taking.
\end{sectionmeta}
\glspl{api} are an instance of information hiding.
By hiding implementation behind an interface, the amount of dependencies in a code base is decreased.  
They allow access to an already implemented code base, such that developers may focus on the specifics of the task at hand rather than writing the entire backend themselves.
In addition, implementation details may be abstracted away when working through an interface,  decreasing the amount of knowledge needed by the developer.
These factors make \glspl{api} a very popular tool in professional development, as they have a positive impact on productivity. 

Today graphics APIs exist at different levels of abstraction. 
\glspl{api} are classified as being either low-level, e.g. OpenGL \& Direct3D, or high-level, e.g. OpenSceneGraph \& Aardvark.
In this report we focus on low-level graphics \glspl{api} as they are often used for performance-intensive applications such as video games.
Furthermore, the high-level \glspl{api} are usually build on top of low-level \glspl{api}.  
Therefore a lot of applications depend on these \glspl{api}APIs in some way.
When it comes to modern low-level graphics \glspl{api}, the interface itself and the specific implementation are handled by separate entities. 
One company designs the \gls{api} and its functionality, and GPU manufacturers can then support interfaces by including implementations of them in their device drivers.

Since the 90s, the most popular low-level graphics \glspl{api} have been Microsoft’s Direct3D, part of the DirectX multimedia API collection, and Khrono’s OpenGL. 
Direct3D, while an open standard, is designed to be used exclusively on the Microsoft Windows platform.
The \gls{api} is partially implemented by Microsoft in a closed common runtime, which communicates with a 3rd party GPU driver through a device driver interface.
On the other hand, OpenGL requires that manufacturers write the entire driver themselves.
This allows OpenGL to be cross-platform, being supported on Windows as well as also Unix-based systems like macOS, iOS, Linux and Android, if the device manufacturer has written a driver. However, looking at Valve Corporations hardware and software survey from 2017, made by querying users of their digital distribution platform steam,  almost 99\% of users ran windows with some version of direct3D installed \cite{steamsurvey}.  \todo[inline]{Find a better source and discuss it}

The technology trajectory of the aforementioned \glspl{api} has been moving in a direction of increasing complexity.
In the 00s, there was a move from a fixed function graphics pipeline to a programmable pipeline.
This meant that developers now had to program their own GPU programs, shaders, to be run as part of the pipeline. 
The addition allowed for more flexibility and supported the rendering of more realistic scenes.
This trend seems to continue with the next generation of \glspl{api}, Khrono’s Vulkan and Microsoft’s DirectX 12. 
Developers are now required to write code formerly implemented as part of the driver. 
This is supposed to allow developers to squeeze more performance out of their applications,  in return of additional complexity. 
In addition, these new APIs claim that they are designed with modern CPU/GPU architectures in mind.  \todo[inline]{Find some source for this for both Vulkan and DX12}
Therefore, they support features such as multi-threaded draw calls from the CPU, which is supported by the trend of multicore CPUs. 
There is also the promise of stateless graphics programming, where the rendering context on the GPU is made fully explicit to developers through \glspl{pso}. 
This stands in contrast to earlier \glspl{api}, where the state is implicit and it can be changed at runtime. 
\glspl{pso} are made at compile time, promising for faster switches between render states.
This is promising news, as state switching is  one of the biggest performance-issues within graphics programming \cite{worister2013lazy}.
\section{Related Works}\label{sec:related_works}
This section will describe the related works within this project.
From these papers it was discovered that a \gls{GPU} is used for two kinds of operation: 
The first purpose for the \gls{GPU} is for rendering graphics, this is used in video games, video rendering and overall graphical visualization in any kind of application. 
Secondly as a GP\gls{GPU}, meaning using the \gls{GPU} for general purpose programing and utilizing its massive parallelism.
The following sections will further explain the related works, and why they are relevant.

\subsection{Graphics}
This section will describe how \glspl{GPU} usually are used in terms of rendering graphics, and elaborate on what kind of issues that can be encountered whilst working with rendering graphics.

Let's Fix OpenGL \cite{fix_opengl} is a article that attempts to expose the shortcommings of OpenGL, and suggests what can be done in order to make OpenGL better. 
The issues mentioned in the article also applies to DirectX according to the author. 
The article identifies six issues: 

\paragraph{1} Programmers must juggle between C/C++ and HLSL and GLSL.
Meaning they have to switch between coding the renderer and coding the shaders in different programming languages.

\paragraph{2} The communication between CPU and \gls{GPU} is brittle. 
They communicate to share data through commands written in the application, there is no way to statically check if the variable name in the application and shader match.
This is a error-prone approach, as such bugs can first be found during runtime of the application.

\paragraph{3} Massive meta-programming. 
There can be generated thousands of varients of a shader program, this is not good as it comes at a performance cost.

\paragraph{4} Different semantics for each shader stage. 
This contributes to the issue of having to keep track of semantics depending on what kind of shader is being worked on.
It makes more difficult to use OpenGL, and increases the learning curve.

\paragraph{5} No type system for vectors when converting between spaces. 
It has to be done manually, it should be possible to simply standardize such an approach.

\paragraph{6} Diffcult to verify correctness of a graphics application. 
There is no way to test whether whatever an application renders, is the actual desired result.

In the conclusion, the article mentions Vulkan and that it might be the solution to the issues of OpenGL. 
Additionally, it also encourages development of new frameworks to rival OpenGL. 
However a shortcomming of the article is that all the listed issues is only based on the authors opinion that he has discussed with his collegues, and not used a proper methodology to collect this data. 
Although they do reveal that OpenGL is not a API without flaws.

The value of this article is that it gives insight as to how a good graphics \gls{API} should work, it's good arguments to take into consideration when trying to evaluate an \gls{API}.

\todo{An Incremental Rendering VM \cite{haaser_2015_incremental}}

\subsection{Comparrison}
There are several options in terms of what \gls{GPU} \gls{API} that can be used when a developer needs to render graphics in an application. This section discusses articles that compare \gls{GPU} \glspl{API} against one another, and why and when certain \glspl{API} should be chosen over others.

\todo{Direct3D 11 vs 12 A Performance Comparison Using Basic Geometry \cite{2016_direct3d} is a master thesis...}

\todo{Reducing Driver Overhead in OpenGL, Direct3D and Mantle  \cite{dobersberger_2015_reducing} is a master thesis...}

Evaluation of multi-threading in Vulkan \cite{blackert_2016_evaluation} is a master thesis that attempts at evaluating the multi-threading performance of Vulkan. It does so by comparing it to its predescor; OpenGL. 
Additionally it also evaluates the programmability of Vulkan. 
In the conclusion, the thesis states that Vulkan can give more throughput than OpenGL.
Not all applications will gain a significant performance boost with using Vulkan over OpenGL. 
This would be in cases where multi-threading is not needed, or if the application is not CPU bound. 
No methodology was used in evaluating the programmability of Vulkan, it is based on the personal experience of the author, which is not a good enough method to evaluate an \gls{API} on. 
It only states that Vulkan is more difficult to work with than OpenGL, because there is more overhead since Vulkan is at a lower abstraction level than OpenGL. 

For future work the thesis encourages to further evaluate the performance of Vulkan multi-threading  capabilities by comparing it to Direct3D 12, and testing the portability of Vulkan on various operating systems and different \gls{GPU} manufactoreres.

\subsection{\gls{GPGPU}}
This section will describe related work as to what GP\gls{GPU} are, what they are used for, and how they relate to the project.

Designing efficient sorting algorithms for Manycore \glspl{GPU} \cite{satish_2009_designing} is an article which describes the development of a custom implemented radix sort and a merge sort, and prove the capabilites of CUDA and \glspl{GPGPU} parallarism, by running several Nvidia \glspl{GPU} for comparrison. 
This article demonstrates how much throughput there is to be gained from \glspl{GPGPU}, and how a comparison can be done. 
A critique of the paper though is that mege sort and radix sort starts to show strange performance spike patterns when working with large workload. 
This is never explained as to why, it should have been elaborated or at the least given a guess.
The article demonstrates how powerful the \gls{GPU} is when it is able to do some things better than a \glspl{CPU}.

The GPU Computing Era \cite{gpu_computing_era} is an article, which discusses the benefits of utilizing \gls{GPU} parallarism to run applications that previously were deemed too time consuming to use in practice. 
The article claims that single-threaded applications no longer perform well enough up against multi-threaded applications, and the industry should adapt to GP\gls{GPU} technology. 
It also mentions, how the \gls{GPU} becomes more powerful all the time by doubling up on its transistors for every 18th month. 
One way of utilizing \gls{GPGPU} is through CUDA. (Nvidia implementation of a gls{GPGPU} gls{API}.)
CUDA programs are very scalable according to the article, and it is thus an excellent tool to utilize \gls{GPGPU} functionality, and encourages more ussage of \gls{GPGPU}.
It is important to note that the authors of this article are from Nvidia, and they are likely biased towards how much power a \gls{GPGPU} can provide, and how good CUDA is. 

This article is revevant as it shows an alternative use of the \gls{GPU}, it is also explains how it achieves the parallarism which contributes to a better \gls{GPU} throughput than a \gls{CPU} can provide.

There have been made a number of tools in order to make utilization of GPGU easier, these Higher Level GP\gls{GPU}'s tool includes the following: Firepile (Scala) \cite{2011_firepile}, OCaml GP\gls{GPU} \cite{bourgoin_2017_high}, PyCuda and PyOpenCL \cite{2012_pycuda_pyopencl} and Chestnut \cite{stromme_2012_chestnut}.
It is important to mention that these \glspl{API} are only some of the tools that provide higher abstraction level GPGPU programming. 
Additionally these tools are mostly academeic experiements.

These tools were made in order to make it easier and less error prone to utilize GP\gls{GPU} for less skilled developers.
As it can be difficult to work with the low level \gls{API} (eg. CUDA).
The value in these articles lies in how the developers expose lower level \glspl{API} to a custom made higher level \gls{API}, and whether or not they are easier to use than their lower-level counterparts. 
Addtionally, it is also interesting to see how well these higher level abstraction compare performance-wise to the lower level ones. 
However, a pitfall the four papers has is that they do not have a proper methodology to test out the programmability of these \glspl{API}, which is highly desired when the goal of these tools is to be easier and more intuitive to use than their lower-level counterparts.
The articles attempts to show the programmability of their \glspl{API} by exemplifying what kind of issues their \gls{API} can solve, the syntax of using the \gls{API} and sometimes with a performance evaulation.
However it all comes down to just the authors opinion.

Debunking the 100x \gls{GPU} vs. \gls{CPU} Myth \cite{lee_2010_debunking} claims that GP\glspl{GPU} are not that much better than \glspl{CPU}. 
It references several papers that claim \glspl{GPU} can be 100 (or more) times better than a \gls{CPU}, and attempts to debunk them. 
With the data that the article collects it concludes that \glspl{GPU} are only 2x times better on average than the \glspl{CPU}. 
The testing was done by writing several algorithms and implement them for both the \gls{GPU} and \gls{CPU}, running the algorithms, observing the performance, and then comparing the results. 
However the \gls{CPU} implementation is highly optimized as the authors and software writers are from Intel, whilst the GP\gls{GPU} implementation for the algorithms is not optimized to its fullest. 
Data from an article like this would have been more meaningful if the GP\gls{GPU} implementation was written some of the best people from Nvidia (since they tested with a Nvidia card) instead of someone from Intel.

\paragraph{}
From these sources it is possible to form a better overview of what is currently going on in the area of \glspl{GPU} in the scientific community. 
Based on these sources, it can be concluded that not many articles are looking into the programmability of \gls{GPU} \glspl{API}. 
The thesis papers were the only once that made an attempt at evaluating Vulkan and DirectX12.

Aditionally there seems to be a lack on papers discussing Direct3D 12 and Vulkan, and should therefore be considered a field that is worthwhile looking into.

\section{Problem Statement}\label{sec:problem_statement}
\begin{sectionmeta}
We recollect and examine the evidence presented in the previous sections.
That evidence is then used to define an area of interest, which the rest of the report is going to be about.
\end{sectionmeta}

\todo[inline]{Summarize evidence from the previous sections.}

\begin{problemstatement}
	Modern graphics APIs, Vulkan & Direct3D 12, are becoming more low level as to get around driver overhead.
	We suspect that this risks sacrificing programmability for graphic performance. 
	To look into this, we will compare the APIs on performance and programmability.
	This comparison should give us an insight into modern APIs as well as lay the groundwork for a new API, which builds on top of either Direct3D 12 or Vulkan.
\end{problemstatement}

\todo[inline]{Transition from problem statement to strategy}

\begin{itemize}
	\item How do Direct3D 12 and Vulkan compare on Performance and Programmability?
	\item How does Direct3D 12 and Vulcan try to improve on graphics processing, and are they succeeding?
	\begin{itemize}
		\item Where are they different and why?
		\item Where are their improvements similar? 
		How does the similar parts of the APIs compare to each other?
	\end{itemize}
	%\item (How would it be possible to abstract DX12/Vulkan to a higher level?)
\end{itemize}