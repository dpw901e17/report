\chapter{Introduction}\label{ch:introduction}
In the early days of consumer electronics, computer visuals were mainly displayed as text.
Yet, with the combined rise of microcomputers and \glspl{GUI} in the 80's, graphics and visuals became a hot topic for both developers and consumers.  
In the 1990s, highly parallel graphics hardware in the form of \glspl{GPU} were introduced to render 3D models in real time.
The first \gls{GPU} was created by Nvidia, its main purpose was to render 3D graphics better and faster than before. \cite{NvidiaFirstGPU} \todo{måske nævne hvilken GPU det er? Kunne diskutere om den nye GPU var så forskellig fra grafikkort fra tidligere}
This piece of hardware has become a stable in modern consumer computers, either as an individual component or integrated the CPU.

%Computers and applications are no longer the niche market it was in its early days, and from that comes a need for intuitive interfaces for users. 
%Around the 1970s the first computers with \glspl{GUI} started to appear to improve the user experience.
To ease development of graphical applications, graphics \glspl{API} were created. 
These allow developers to work with the GPU at a more abstract level, through an interface.
%But to develop these \glspl{GUI}, developers use a lot of time working with the graphics \glspl{API}.
Developers have used these \glspl{API} to create impressive applications, such as modern video games, which render graphics in real time at 60 frames per second (at times more than that), as well as the graphically demanding \glspl{GUI} of modern operating systems.

%As hardware improves, so must the \gls{API} to better use the improvements.
%Another function of the \gls{API} is to make it easier for the developers to use the hardware so improvement in that area is also interesting for the \gls{API} developers.
Graphics \glspl{API} have to evolve alongside modern CPU and GPU architectures, to take advantage of new features. 
Yet, academic research into the evaluation of \glspl{API} and how to improve them is limited.
Even if the newer \glspl{API} supports operations that improve performance, if the feature is too difficult to use most developers will ignore it.

Video games have always had high demands on hardware, even before the introduction of \glspl{GPU}.
Because of the constant demand from developers and consumers to render more demanding graphics in real time, \gls{GPU} manufacturers have continually made the \gls{GPU} throughput higher with each new \gls{GPU} release

With the release of the most recent graphics \glspl{API}; Vulkan and Direct3D 12, there has been a push towards a more low-level API design, which gives the programmer more control over the graphics pipeline.
Much of the functionality that was previously handled by the driver, should now be handled by application developers. 
This has some advantages, which will be discussed in detail in the following chapter \cref{sec:graphics_apis}.
Yet, the argument that lower levels of abstraction results in better \glspl{API} would suggest that writing in binary or assembly would provide the best \glspl{API}.\todo{Synes at der skal gås mere i detalje med den her linje, eller vi skal vente med den diskution til senere}

In this report, we are instested in how the newer \glspl{API} lower their abstraction-level, and what advantages and disadvantages this approach has both on performance and programability.
The aim is to generate insight into the graphics \glspl{API}; Direct3D 12 and Vulkan, and to generate tools for future developers that can be used to evaluate and improve their \glspl{API}.\todo{Hvad er det for nogle tools vi taler om her?}

Direct3D 12 and Vulkan are chosen, because they are released close to eachother (2015 and 2016 respecively), and have some overlapping ideas on how to improve graphics \glspl{API}, such as better multithread support on the CPU.
The predesessors to these \glspl{API} (Direct3D 11 and OpenGL 4.5) are some of the most broadly used graphics \glspl{API} for realtime rendering, and we suspect that Direct3D 12 and Vulkan will have similar success in the comming years.

In the following section we will define the problem statement of the report in detail.
The reply to the questions presented will be the goal of the rest of the rest of the work presented here.

\section{Problem Statement}\label{sec:problem_statement}
\begin{sectionmeta}
We recollect and examine the evidence presented in the previous sections.
That evidence is then used to define an area of interest, which the rest of the report is going to be about.
\end{sectionmeta}

\todo[inline]{Summarize evidence from the previous sections.}

\begin{problemstatement}
	Modern graphics APIs, Vulkan & Direct3D 12, are becoming more low level as to get around driver overhead.
	We suspect that this risks sacrificing programmability for graphic performance. 
	To look into this, we will compare the APIs on performance and programmability.
	This comparison should give us an insight into modern APIs as well as lay the groundwork for a new API, which builds on top of either Direct3D 12 or Vulkan.
\end{problemstatement}

\todo[inline]{Transition from problem statement to strategy}

\begin{itemize}
	\item How do Direct3D 12 and Vulkan compare on Performance and Programmability?
	\item How does Direct3D 12 and Vulcan try to improve on graphics processing, and are they succeeding?
	\begin{itemize}
		\item Where are they different and why?
		\item Where are their improvements similar? 
		How does the similar parts of the APIs compare to each other?
	\end{itemize}
	%\item (How would it be possible to abstract DX12/Vulkan to a higher level?)
\end{itemize}