\section{Graphics \acrshort{API}}\label{sec:graphics_apis}
\begin{sectionmeta}
In this section we should introduce, why APIs are important to use, the basics on what graphics APIs are and finally what direction moderns APIs are taking.
\end{sectionmeta}
\glspl{API} are an instance of information hiding.
By hiding implementation behind an interface, the amount of dependencies in a code base is decreased.  
They allow access to an already implemented code base, such that developers may focus on the specifics of the task at hand rather than writing the entire backend themselves.
In addition, implementation details may be abstracted away when working through an interface,  decreasing the amount of knowledge needed by the developer.
These factors make \glspl{API} a very popular tool in professional development, as they have a positive impact on productivity. 

Today graphics \glspl{API} exist at different levels of abstraction. 
\glspl{API} are classified as being either low-level, e.g. OpenGL \& Direct3D, or high-level, e.g. OpenSceneGraph \& Aardvark.
In this report we focus on low-level graphics \glspl{API} as they are often used for performance-intensive applications such as video games.
Furthermore, the high-level \glspl{API} are usually build on top of low-level \glspl{API}.  
Therefore a lot of applications depend on these \glspl{API} in some way.
When it comes to modern low-level graphics \glspl{API}, the interface itself and the specific implementation are handled by separate entities. 
One company designs the \gls{API} and its functionality, and \gls{GPU} manufacturers can then support interfaces by including implementations of them in their device drivers.

Since the 90s, the most popular low-level graphics \glspl{API} have been Microsoft’s Direct3D, part of the DirectX multimedia \gls{API} collection, and Khrono’s OpenGL. 
Direct3D, while an open standard, is designed to be used exclusively on the Microsoft Windows platform.
The \gls{API} is partially implemented by Microsoft in a closed common runtime, which communicates with a 3rd party \gls{GPU} driver through a device driver interface.
On the other hand, OpenGL requires that manufacturers write the entire driver themselves.
This allows OpenGL to be cross-platform, being supported on Windows as well as also Unix-based systems like macOS, iOS, Linux and Android, if the device manufacturer has written a driver.
However, looking at Valve Corporations hardware and software survey from 2017\cite{steamsurvey}, made by querying users of their digital distribution platform Steam,  almost 99\% of users ran Windows with some version of Direct3D installed. 
%\todo[inline]{Find a better source and discuss it}

The technology trajectory of the aforementioned \glspl{API} has been moving in a direction of increasing complexity.
In the 00s, there was a move from a fixed function graphics pipeline to a programmable pipeline.
This meant that developers now had to program their own GPU programs, shaders, to be run as part of the pipeline. 
The addition allowed for more flexibility and supported the rendering of more realistic scenes.
This trend seems to continue with the next generation of \glspl{API}, Khrono’s Vulkan and Microsoft’s DirectX 12. 
Developers are now required to write code that was formerly implemented as part of the driver. 
This is supposed to allow developers to squeeze more performance out of hardware,  in return of additional complexity. 
In addition, these new \glspl{API} claim that they are designed with modern \gls{CPU}/\gls{GPU} architectures in mind.  \todo[inline]{Find some source for this for both Vulkan and DX12}
Therefore, they support features such as multi-threaded draw calls from the \gls{CPU}, which is supported by the trend of multicore \glspl{CPU}. 
There is also the promise of stateless graphics programming, where the rendering context on the \gls{GPU} is made fully explicit to developers through \glspl{PSO}. 
This stands in contrast to earlier \glspl{API}, where the state is implicit and it can be changed at runtime. 
\glspl{PSO} are made at compile time, promising for faster switches between render states.
This is promising news, as state switching is one of the biggest performance-issues within graphics programming \cite{worister2013lazy}.
